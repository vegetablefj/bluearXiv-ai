\documentclass[9pt, a4paper]{article}
\usepackage{geometry}        
\geometry{scale=0.75}
\usepackage{setspace}        
\usepackage{fontspec}        
\usepackage{xeCJK}           
\usepackage{ctex}            
\usepackage{CJKutf8}         
\usepackage{zhnumber}     
%\usepackage[english,chinese]{babel}
\usepackage{polyglossia}

\usepackage{amsmath}         
\usepackage{amssymb}         
\usepackage{amsthm}          
\usepackage{mathtools}       
\usepackage{mathrsfs}        
\usepackage{latexsym}        
\usepackage{stmaryrd}        

\usepackage{graphicx}        
\usepackage{tikz}            
\usepackage{tikz-cd}                   

\usepackage[authoryear]{natbib} 
\usepackage{etoolbox}        

\usepackage{xcolor}          
\usepackage{hyperref}        
\hypersetup{                 
    colorlinks=true,         
    citecolor=green,         
    urlcolor=red,            
    bookmarksnumbered=true   
}
%\usepackage{cleveref}           
\usepackage{enumitem}        
\usepackage{fancyhdr}        
\usepackage{lmodern}         

\usepackage{pdfpages} 

\theoremstyle{plain}
\newtheorem{theorem}{定理}
\newtheorem{lemma}[theorem]{引理}
\newtheorem{corollary}[theorem]{推论}
\newtheorem{proposition}[theorem]{命题}

\theoremstyle{definition}
\newtheorem{definition}[theorem]{定义}
\newtheorem{example}[theorem]{示例}
\newtheorem{remark}[theorem]{注记}

\theoremstyle{remark}

\definecolor{arxivblue}{RGB}{0, 102, 204}
\newcommand{\arxivwithtarget}[3]{\hypertarget{#1}{}\href{https://arxiv.org/abs/#1}{#1}, \textbf{\textcolor{arxivblue}{#2}}, #3. }
\newcommand{\arxiv}[3]{\href{https://arxiv.org/abs/#1}{#1}, \textbf{\textcolor{arxivblue}{#2}}, #3. }



\begin{document}
\begin{center}
    \Huge\textbf{不漏arXiv:\today \footnote{本文档由...自动生成. 实际抓取的是new页面的数据, 即最近的有所在分类论文变动的一天的数据. This document is automatically generated by ... The data actually captured is from the "new" page, that is, the data of the most recent day when there were changes in corresponding subjects.}\footnote{感谢arXiv提供的服务. Thanks for services prodived by arXiv.}\footnote{评论和精选由AI生成, 不代表任何人对论文本身的看法. 精选依赖于论文与给定关键词的匹配度. Comments and selection of good papers are generated by AI, not showing anyone's point of view about those papers. The selection also depends on the matched-degrees between papers and given keywords.}\footnote{下面的计数基于主学科, 不计重数. The following counters are based on main subject, not counting multiplicities.}} 
\end{center}

\begin{center}
    %counter_begin
math.AG: 11

math.RT: 7

math.QA: 1

others: 7

%counter_end   
\end{center}

\setcounter{section}{-1}
\section{精选 Selections}
    %selection_begin
\arxiv{2601.21052}{Non-abelian Rees construction and pure motives}{Yves André}\textbf{math.AG}, math.CT, math.NT. \hyperlink{2601.21052}{$\rightsquigarrow$}

\arxiv{2601.21871}{Polync varieties and multiparameter Kulikov models}{Philip Engel}\textbf{math.AG}, math.SG. \hyperlink{2601.21871}{$\rightsquigarrow$}

\arxiv{2601.22085}{$K$-Equivalence and Integral Cohomology}{Matthew Satriano, Evan Sundbo}\textbf{math.AG}. \hyperlink{2601.22085}{$\rightsquigarrow$}

\arxiv{2601.21496}{Log-concavity and unimodality of cluster monomials of type $A_3$}{Zhichao Chen}\textbf{math.RT}, math.CO, math.RA. \hyperlink{2601.21496}{$\rightsquigarrow$}

%selection_end

    %body_begin
\section{math.AG}
\arxivwithtarget{2601.21052}{Non-abelian Rees construction and pure motives}{Yves André}\textbf{math.AG}, math.CT, math.NT.

本文利用非交换Rees构造, 建立了拟齐次空间与特定范畴之间的Galois对应关系, 并将其应用于动机范畴, 特别是在代数周期方面提供了新的证明和推广, 涉及Clozel-Deligne定理关于有限域上abelian簇上的数值等价性.


\arxivwithtarget{2601.21304}{Real gamma distribution on analytic bundles of flag varieties}{Haoming Wang}\textbf{math.AG}, math.CO, math.PR.

本文引入了四种矩阵正态分布, 扩展了可变级别的分离协方差$\varPhi \otimes \varPsi$, 并考虑了样本方差和协方差的联合分布, 从而得到了产品矩分布. 通过将这些结果应用于旗流形, 分类了双旗和单旗.


\arxivwithtarget{2601.21456}{Classification of low degree del Pezzo orbifolds}{Saptarshi Dandapat}\textbf{math.AG}.

本文用adjunction formula, Riemann-Roch theorem, Hodge Index theorem和Kawamata-Viehweg vanishing theorem证明了低度del Pezzo orbifolds的分类, 为Campana orbifolds的研究提供了基础贡献.


\arxivwithtarget{2601.21466}{A note on irreducible slice algebraic sets}{Anna Gori, Giulia Sarfatti, Fabio Vlacci}\textbf{math.AG}.

本文证明了如果$I$是环中四元数切正规多项式的右根理想和拟主理想, 则共零集$V_c(I)$的对称化$\mathbb{S}_{V_c(I)}$是不可约代数集. 结合此结论与之前论文[3]的结果, 得到了对于$I$为根理想的情况, $V_c(I)$不可约当且仅当$I$为拟主理想的结论.


\arxivwithtarget{2601.21559}{New examples of non-Fourier-Mukai exact functors via non-isomorphic octahedra}{Alberto Canonaco, Mattia Ornaghi}\textbf{math.AG}, math.CT, math.RT.

本文研究了一个三角范畴$\mathscr S$, 该范畴包含一个三个对象的完整且强异常序列, 每个对象之间的Hom空间是一维的. 通过探讨$\mathscr S$到另一个三角范畴$\mathscr T$的精确函子的同构类与其在$\mathscr T$中满足自然条件的八面体同构类之间的关系, 作者构造了一个从$\mathscr S$到$\mathbf D^b(\Bbbk[x]\text{-}\mathrm{mod})$的精确函子, 该函子不具有dg提升. 这一结果提供了非Fourier-Mukai精确函子的一个明确实例, 特别是在$\mathbf D^b(\mathbb P^2)$和$\mathbf D^b(\mathbb P^1)$之间.


\arxivwithtarget{2601.21871}{Polync varieties and multiparameter Kulikov models}{Philip Engel}\textbf{math.AG}, math.SG.

本文研究了局部表现为正常交叉奇异性的多项nc奇异性的"polync品种". 通过引入d-半稳定性并结合最近关于对数Bogomolov-Tian-Todorov定理的突破性工作, 本文解决了d-半稳定、K平凡的polync品种的可平滑性问题, 并将Kulikov模型的组合描述推广到多参数基的情形.


\arxivwithtarget{2601.21958}{Linear systems on rational surfaces}{Cyril J. Jacob, Ronnie Sebastian}\textbf{math.AG}.

本文针对Hirzebruch表面提出了类似于$\mathbb{P}^2$上SHGH猜想的猜想, 并证明了当在非常一般的位置上吹嘘$\mathbb{F}_e$时, 该猜想对于$r\leqslant e+4$个点成立.


\arxivwithtarget{2601.21973}{Generalizations of tropical Tevelev degrees}{Erin Dawson}\textbf{math.AG}, math.CO.

本文研究了来自特定曲线模空间映射的广义热带Tevelev度量, 引入了一个额外的整数参数$\ell$, 使得曲线度和标记点数量分别为$d = g + 1 + \ell$和$n = g + 3 + 2\ell$. 这些结果扩展了热带Tevelev度量的计算和结构模式, 并揭示了超越经典设置的新行为.


\arxivwithtarget{2601.22004}{Highest weight categories via pairs of dual exceptional sequences}{Agnieszka Bodzenta, Alexey Bondal}\textbf{math.AG}, math.RT.

本文通过一对对偶异常序列给出了判别准则, 证明了某些几何起源的阿贝尔范畴是最高权重范畴, 并且提供了新的证明方法.


\arxivwithtarget{2601.22072}{On singularities of determinantal hypersurfaces}{Daniel Bath, Mircea Mustaţă}\textbf{math.AG}.

本文考虑了由光滑簇$X$上的$s\times r$正则函数矩阵的最大余子式定义的闭子方案$Z$对应的 incidence 对应$W$在$Y=X\times \mathbf{P}^{r-1}$中的性质, 研究了$(X,Z)$和$(Y,W)$的 log canonical 值之间的关系. 特别地, 当$r=s$时, 证明了$Z$有理性奇点当且仅当$W$在$Y$中有纯维$r$且有理性奇点.


\arxivwithtarget{2601.22085}{$K$-Equivalence and Integral Cohomology}{Matthew Satriano, Evan Sundbo}\textbf{math.AG}.

本文引入了积分Hodge多项式的积分版本, 证明其在代数簇的Grothendieck环上定义良好, 并且$K$-等价的光滑射影簇具有同构的积分上同调群.


\section{math.RT}
\arxivwithtarget{2601.21156}{Further results on fuzzy negations and implications induced by fuzzy conjunctions and disjunctions}{Xin-Tong Zhang, Xue-ping Wang}\textbf{math.RT}.

本文深入研究了由模糊合取和析取诱导的模糊否定的一些性质, 并将其应用于模糊否定的特征化. 进一步利用获得的模糊否定特征化, 探讨了由模糊析取和否定生成的$(D,N)$-蕴涵的一些性质. 最后描述了由模糊析取和否定生成的$(D,N)$-蕴涵.


\arxivwithtarget{2601.21197}{A family of simple $U(\mathfrak{h})$-free modules of rank 2 over $\mathfrak{sl} (2)$}{Dimitar Grantcharov, Khoa Nguyen, Kaiming Zhao}\textbf{math.RT}.

本文研究了$\mathfrak{sl}(2)$-模在$\mathbb{C}$上的自由有限秩$U(\mathfrak{h})$-模, 并给出了秩为2的标量型简单模的显式分类以及它们是否同构的判别准则. 这两者归结为$\mbox{GL}_2({\mathbb C}[h])$的扭曲共轭类, 并依赖于Cohn的标准形式.


\arxivwithtarget{2601.21496}{Log-concavity and unimodality of cluster monomials of type $A_3$}{Zhichao Chen}\textbf{math.RT}, math.CO, math.RA.

本文用严格的方法证明了类型$A_3$的簇多项式的对数凹性和单峰性, 为高秩簇多项式猜想提供了重要进展.


\arxivwithtarget{2601.21614}{Admissible modules over affine Lie superalgebras: The final step in the characterization}{Malihe Yousofzadeh}\textbf{math.RT}.

本文研究了零级模块, 并完成了对仿射李超代数简单可接受模的分类.


\arxivwithtarget{2601.21834}{Reduction theorems for a conjecture on basis in source algebras of blocks of finite groups}{Tiberiu Coconet, Constantin-Cosmin Todea}\textbf{math.RT}, math.GR.

本文用reduction theorems证明了关于Barker和Gelvin猜想的部分结果, 即任何有限群块的源代数的单位群包含一个由缺陷群稳定化的基, 为该问题提供了新的见解.


\arxivwithtarget{2601.21850}{Center of the affine $\mathfrak{gl}_{n|1}$ at the critical level and pseudo-differential operators}{Dražen Adamović, Boris Feigin, Shigenori Nakatsuka}\textbf{math.RT}, math-ph, math.QA.

本文证明了在临界级数下, affine $\mathfrak{gl}_{n|1}$ 的中心由特定伪微分算子在Cartan子代数取值时的系数生成, 这是有限情形Harish-Chandra同构的仿射类比. 基于此, 我们推导出了中心的特征公式, 该公式与带有坑条件的平面分区的生成函数一致.


\arxivwithtarget{2601.21954}{Asymptotic Expansion for Expanding Spherical Averages in Real Rank One}{Zhiyuan Deng, Yutian Sun}\textbf{math.RT}.

本文用harmonic analysis和representation theory工具证明了关于expanding non-spherical averages在实秩一李群紧商空间上的精确渐近公式, 为相关分析提供了贡献.


\section{math.QA}
\arxivwithtarget{2601.22017}{Fully exact and fully dualizable module categories}{Azat M. Gainutdinov, Robert Laugwitz}\textbf{math.QA}, math.CT, math.RT.

本文定义了完全精确模范畴, 这是有限编织张量范畴的精确模范畴的一个子类, 且在相对德尔米纳乘积下稳定. 通过具体例子展示了精确模范畴在这个乘积下并不稳定, 并观察到完全精确模范畴在精确模范畴类中形成稠密子集. 文章还探讨了当模范畴是完美时的情况, 并给出了具体的例子进行分析.


\section{others}
\arxivwithtarget{2601.20936}{Quiver-Invariant Dualities between Brane Tilings}{Minsung Kho, Seong-Jin Lee, Rak-Kyeong Seong}\textbf{hep-th}, math-ph, math.AG.

本文研究了共享同一quiver但超势能不同的4d N=1超对称规范理论之间的对应关系, 通过brane tiling中的单个`tilting`突变实现, 这等价于quiver中特定节点的一系列Seiberg对偶变换.


\arxivwithtarget{2601.20960}{On the Integrable Structure of the SU(2) Wess-Zumino-Novikov-Witten Model}{Lacroix Sylvain, Molines Adrien}\textbf{hep-th}, math-ph, math.QA, math.RT.

本文研究了SU(2) Wess-Zumino-Novikov-Witten模型的量子可积结构, 通过直接对角化和ODE/IQFT猜想验证了局部算子的谱, 展示了两者的一致性.


\arxivwithtarget{2601.21195}{$q$-deformations of the Tsetlin library}{Arvind Ayyer, Sarah Brauner, Jan de Gier, Anne Schilling}\textbf{math.CO}, math.PR, math.RT.

本文定义了$q$-变形的Tsetlin图书馆, 并通过将其与有限域向量空间$\mathbb{F}_q^n$上完整旗的马尔可夫链相关联, 计算了其平稳分布和谱. 此外, 还推广了$q$-Tsetlin图书馆到单词(包含重复字母) , 并计算了其平稳分布和谱.


\arxivwithtarget{2601.21588}{Explicit Construction of Maass Wave Forms and Their Petersson Inner Products}{Daichi Tanaka}\textbf{math.NT}, math.RT.

本文用explicit construction方法证明了Maass波形式及其Petersson内积的显式计算, 为Hecke字符上的实二次域提供了新的Maass波形式.


\arxivwithtarget{2601.21645}{Identifiable Equivariant Networks are Layerwise Equivariant}{Vahid Shahverdi, Giovanni Luca Marchetti, Georg Bökman, Kathlén Kohn}\textbf{cs.LG}, math.CT, math.RT.

本文探讨了端到端等变性和分层等变性之间的关系, 并证明了一种网络在其参数选择适当的情况下, 其各层可以是某些群作用下的等变结构. 这一结果假设模型参数在某种意义上是可识别的, 而这种可识别性已在文献中被广泛研究和验证.


\arxivwithtarget{2601.21888}{Synchronization points: growth, asymptotics, congruences, and the synchronization zeta function}{Alexander Fel'shtyn, Mateusz Slomiany}\textbf{math.DS}, math-ph, math.AT, math.GR, math.RT.

本文引入了同步点同步ζ函数, 并研究了其性质. 此外, 定义了同步点的增长率, 并在紧致连通阿贝尔群自同态的背景下推导出了显式公式.


\arxivwithtarget{2601.22089}{On set-theoretic solutions of pentagon equation and positive basis Hopf algebras}{Ilaria Colazzo, Geoffrey Janssens}\textbf{math.RA}, math.QA.

本文研究了集论解五边形方程与Hopf代数之间的联系, 证明了有限解对应于具有正基性质的Hopf代数, 并通过傅里叶变换给出了群代数$k[G]$产生集论解的所有基的描述.


%body_end
    
    %others_begin

    %others_end
\end{document}