\documentclass[9pt, a4paper]{article}
\usepackage{geometry}        
\geometry{scale=0.75}
\usepackage{setspace}        
%\usepackage{fontspec}        
\usepackage{xeCJK}           
\usepackage{ctex}            
%\usepackage{CJKutf8}         
\usepackage{zhnumber}     
%\usepackage[english,chinese]{babel}
%\usepackage{polyglossia}

\usepackage{amsmath}         
\usepackage{amssymb}         
\usepackage{amsthm}          
\usepackage{mathtools}       
\usepackage{mathrsfs}        
\usepackage{latexsym}        
\usepackage{stmaryrd}        

\usepackage{graphicx}        
\usepackage{tikz}            
\usepackage{tikz-cd}                   

%\usepackage[authoryear]{natbib} 
%\usepackage{etoolbox}        

\usepackage{xcolor}          
\usepackage{hyperref}        
\hypersetup{                 
    colorlinks=true,         
    citecolor=green,         
    urlcolor=red,            
    bookmarksnumbered=true   
}
%\usepackage{cleveref}           
%\usepackage{enumitem}        
%\usepackage{fancyhdr}        
%\usepackage{lmodern}         

%\usepackage{pdfpages} 

\theoremstyle{plain}
\newtheorem{theorem}{定理}
\newtheorem{lemma}[theorem]{引理}
\newtheorem{corollary}[theorem]{推论}
\newtheorem{proposition}[theorem]{命题}

\theoremstyle{definition}
\newtheorem{definition}[theorem]{定义}
\newtheorem{example}[theorem]{示例}
\newtheorem{remark}[theorem]{注记}

\theoremstyle{remark}

\definecolor{arxivblue}{RGB}{0, 102, 204}
\newcommand{\arxivwithtarget}[3]{\hypertarget{#1}{}\href{https://arxiv.org/abs/#1}{#1}, \textbf{\textcolor{arxivblue}{#2}}, #3. }
\newcommand{\arxiv}[3]{\href{https://arxiv.org/abs/#1}{#1}, \textbf{\textcolor{arxivblue}{#2}}, #3. }



\begin{document}
\begin{center}
    \Huge\textbf{不漏arXiv:\today \footnote{本文档由\href{https://github.com/vegetablefj/bluearXiv-ai}{bluearXiv-ai}自动生成. 实际抓取的是``new''页面的数据, 即最近的有所在分类论文变动的一天的数据. This document is automatically generated by \href{https://github.com/vegetablefj/bluearXiv-ai}{bluearXiv-ai}. The data actually captured is from the ``new'' page, that is, the data of the most recent day when there were changes in corresponding subjects.}\footnote{感谢arXiv提供的服务. Thanks for services prodived by arXiv.}\footnote{评论和精选由AI生成, 不代表任何人对论文本身的看法. 精选依赖于论文与给定关键词的匹配度. Comments and selection of good papers are generated by AI, not showing anyone's point of view about those papers. The selection also depends on the matched-degrees between papers and given keywords.}\footnote{下面的计数基于主学科, 不计重数. The following counters are based on main subject, not counting multiplicities.}} 
\end{center}

\begin{center}
    %counter_begin
math.AG: 14

math.RT: 5

math.QA: 4

others: 6

%counter_end   
\end{center}

\setcounter{section}{-1}
\section{精选 Selections}
    %selection_begin
%selection_end

    %body_begin
\section{math.AG}
\arxivwithtarget{2602.16103}{Topology of the Vakil--Zinger moduli space}{Terry Dekun Song}\textbf{math.AG}, math.AT.

本文研究了Vakil--Zinger构造的模空间$\widetilde{\mathcal{M}}_{1,n}(\mathbb{P}^r,d)$的有理同调群, 给出了其生成元并定性描述了生成元间的关系. 方法基于热带几何给出的分层结构以及对Borel--Moore同调权滤过的分析, 并将先前对$\overline{\mathcal{M}}_{g,n}$的研究技术进行了扩展.


\arxivwithtarget{2602.16115}{Algebraic and analytic structure of Morikawa's sangaku problem}{David Krumm}\textbf{math.AG}, math.HO.

本文研究了 Morikawa 的算额问题中内接正方形最小边长的函数 $\mu(r)$ 的性质. 证明了 $\mu$ 是一个代数函数, 在区间 $[1,\infty)$ 上除有限个可计算点外是实解析的, 并给出了其在 $r=1$ 处的泰勒展开计算示例.


\arxivwithtarget{2602.16168}{Proof of Miyanishi's conjecture on endomorphisms of varieties}{Supravat Sarkar}\textbf{math.AG}.

本文证明了Miyanishi猜想, 即对于域$k$上的拟射影簇$X$, 若双有理自同态$\phi$在余维数至少为2的闭子集外是单射, 则$\phi$必为自同构. 该结果推广了Ax的经典定理.


\arxivwithtarget{2602.16185}{On the Enestrom-Kakeya Theorem for polynomials of an octonionic variable}{Ting Yang, Xinyuan Dou}\textbf{math.AG}.

本文研究了八元数多项式零点的分布问题, 将经典的 Enestrom-Kakeya 定理推广到八元数情形. 对于系数非负且单调的多项式, 证明了其零点集包含在八元数空间的闭球内, 并对系数模长或实部单调的情形也得到了一些结果.


\arxivwithtarget{2602.16384}{The jet schemes of the nilpotent cone of $\mathfrak{gl}_n$ over $\mathbb{F}_\ell$ and analytic properties of the Chevalley map}{Avraham Aizenbud, Dmitry Gourevitch, David Kazhdan, Eitan Sayag}\textbf{math.AG}, math.RT.

本文研究了正特征域上幂零矩阵簇(及相关簇) 的jet schemes的维数上界. 该结果应用于Chevalley映射(将矩阵映为其特征多项式) 的解析性质, 并证明了在某些条件下, 该映射将光滑紧支测度推前为具有$L^t$密度函数的测度.


\arxivwithtarget{2602.16397}{Invertible top form on the Hilbert scheme of a plane in positive characteristic}{Avraham Aizenbud, Dmitry Gourevitch, David Kazhdan, Eitan Sayag}\textbf{math.AG}.

本文证明了在正特征域上, 平面Hilbert scheme上存在一个可逆的最高阶微分形式. 这一结果推广了特征零情形下已知的更强结论, 并由此推导出平面对称幂的某些可积性性质.


\arxivwithtarget{2602.16398}{Effective local differential topology of algebraic varieties over local fields of positive characteristics}{Avraham Aizenbud, Dmitry Gourevitch, David Kazhdan, Eitan Sayag}\textbf{math.AG}.

本文为研究局部域上代数簇的定量性质提供了一个框架, 重点关注 $\mathbb{F}_\ell((t))$ 型局部域及其有限扩张. 在此框架下, 证明了隐函数定理等局部微分拓扑标准结果的类比, 并研究了光滑测度在淹没映射下的前推行为.


\arxivwithtarget{2602.16414}{Positive Charts of Toric Varieties}{Veronica Calvo Cortes, Simon Telen}\textbf{math.AG}, hep-th, math.CO.

本文为光滑射影 toric variety 构造了包含其非负点的仿射图, 并证明了这些“正图”源于 nef cone 的光滑子锥. 文章将正几何中的 $u$-方程理论置于一个 toric 框架之下.


\arxivwithtarget{2602.16434}{Logarithmic Hurwitz Spaces in Mixed and Positive Characteristic with Wild Ramification}{Matthias Hippold}\textbf{math.AG}.

本文在混合特征与正特征情形下引入了新的对数Hurwitz空间, 用于参数化具有野分歧的曲线覆盖. 作者证明了这些模空间在某些初始情形下是对数光滑的.


\arxivwithtarget{2602.16492}{Terminalizations of quotients of Fano varieties of lines on cubic fourfolds}{Enrica Mazzon}\textbf{math.AG}.

本文研究了光滑 cubic fourfold 上 lines 的 Fano variety 在有限 symplectic automorphism group 作用下的商簇的 projective terminalization 的分类问题. 作者计算了这些簇的 regular locus 的第二 Betti number 和基本群, 并由此识别出两个新的四维不可约 holomorphic symplectic variety 的形变类.


\arxivwithtarget{2602.16510}{Some rational subvarieties of moduli spaces of stable vector bundles}{Sonia Brivio, Federico Fallucca, Filippo F. Favale}\textbf{math.AG}.

本文构造了具有固定行列式和秩的$\mu_H$-稳定向量丛族, 这些丛由$r+1$个整体截面生成, 并以Grassmann簇为参数空间. 这给出了模空间中与Grassmann簇双有理的特殊子簇.


\arxivwithtarget{2602.16580}{On the Coupled Cluster Doubles Truncation Variety of Four Electrons}{Fabian M. Faulstich, Vincenzo Galgano, Elke Neuhaus, Irem Portakal}\textbf{math.AG}, physics.chem-ph, quant-ph.

本文研究了四电子体系下耦合簇双激发截断簇(CCD)的代数几何性质. 通过理论和数值分析, 确定了该簇在特定轨道数下的次数, 并揭示了其定义关系中的Pfaffian结构.


\arxivwithtarget{2602.16599}{Level structures on cyclic covers of $\mathbb{P}^n$ and the homology of Fermat hypersurfaces}{Eduard Looijenga}\textbf{math.AG}.

本文研究了$\mathbb{P}^n$上沿光滑超曲面$Z'$完全分歧的$\mu_d$-覆盖$Z$, 并建立了$Z'$与$Z$的primitive cohomology上full level $d$结构之间的联系. 在$d=n=3$的特殊情形下, 这一联系为Beauville提出的关于三次曲面与三次超曲面标记的问题提供了答案.


\arxivwithtarget{2602.16685}{Generalized determinantal representation of hypersurfaces}{A. El Mazouni, D. S. Nagaraj, Supravat Sarkar}\textbf{math.AG}.

本文推广了hypersurface的determinantal representation概念, 将其扩展到向量丛determinant line bundle的截面的determinantal representation. 作者给出了若干例子, 并证明了一些存在性的必要条件. 作为应用, 他们构造了$\mathbb{P}^2$上的一族秩为2的不可分解向量丛$E_d$, 使得几乎所有$d$次曲线都能(在$\mathbb{P}^2$的自同构意义下)实现为$E_d$的两个全纯截面的退化轨迹, 并由此得到了一个线性代数应用.


\section{math.RT}
\arxivwithtarget{2601.10934}{Invariant Algebraic $D$-Modules on Connected Reductive Groups}{Rudrendra Kashyap, Ruoxi Li}\textbf{math.RT}, math.AG.

本文研究了复仿射代数群上有限秩左平移不变的代数$D$-模. 通过将其描述为平凡向量丛上的左不变平坦代数联络模去代数规范变换, 作者将分类问题重述为一个显式的常值联络模空间问题. 主要结果针对半单群、一般线性群以及更一般的连通约化群进行了证明.


\arxivwithtarget{2602.15982}{$G_2$ representations and semistandard tableaux}{William M. McGovern}\textbf{math.RT}.

本文利用 semistandard tableaux 实现了 $G_2$ 型复群的不可约有限维表示, 并给出了旗簇定义理想的显式生成元.


\arxivwithtarget{2602.16332}{The invariance of the Auslander-Reiten Formula for hereditary algebras}{Andrew Hubery}\textbf{math.RT}.

本文证明了有限维hereditary algebra的Auslander-Reiten公式在Auslander-Reiten变换下是不变的.


\arxivwithtarget{2602.16389}{On Harish-Chandra's integrability theorem in positive characteristic}{Avraham Aizenbud, Dmitry Gourevitch, David Kazhdan, Eitan Sayag, Itay Glazer, Yotam Hendel}\textbf{math.RT}.

本文研究了正特征域上 $GL_n(F)$ 的尖点表示的分布特征的可积性. 在假设某些代数簇在正特征下存在消奇解的条件下, 证明了可积性定理成立, 并且在特征大于 $n/2$ 时无条件地建立了此类特征的正则性.


\arxivwithtarget{2602.16393}{Orbital integral bounds the character for cuspidal representations of $GL_n(\mathbb{F}_{\ell}((t)))$}{Avraham Aizenbud, Dmitry Gourevitch, David Kazhdan, Eitan Sayag}\textbf{math.RT}.

本文证明了$GL_n(\mathbb{F}_{\ell}((t)))$的不可约尖表示的特征标可以被该表示的矩阵系数的轨道积分以对数因子局部控制. 这一结果在特征0情形是Harish-Chandra可积性定理证明的一部分.


\section{math.QA}
\arxivwithtarget{2602.16017}{Homotopy Lie algebras and coherent infinitesimal 2-braidings}{Cameron Kemp}\textbf{math.QA}, math-ph, math.CT, math.RT.

本文研究了 homotopy Lie algebra ($L_\infty$-algebra) 的表示范畴及其上的高阶结构. 具体地, 作者将 Lada-Markl 模构成的对称幺半 dg-范畴 (symmetric monoidal dg-category) 与相应的 Chevalley-Eilenberg 代数的半自由 dg-模范畴建立了对称幺半 dg-等价.


\arxivwithtarget{2602.16199}{Dual partially harmonic tensors and quantized Schur--Weyl duality}{Pei Wang, Zhankui Xiao}\textbf{math.QA}, math.RT.

本文研究了 Birman-Murakami-Wenzl algebra 的商代数与量子群 $U_q(\mathfrak{sp}_{2m})$ 在某个张量空间上的自同态代数之间的关系. 通过使用 framed tangle 的图范畴和典范基, 证明了从前者到后者的自然同态总是满射.


\arxivwithtarget{2602.16373}{Projective corepresentations and cohomology of compact quantum groups}{Debashish Goswami, Kiran Maity}\textbf{math.QA}, math.OA.

本文研究了compact quantum groups的projective unitary coreresentations及其相关的二阶上同调理论. 作者引入了多种projective corepresentations并研究了它们的提升性质, 同时定义了一个与量子群相关的离散群$\Gamma_\q$, 作为二阶群上同调的一种推广.


\arxivwithtarget{2602.16593}{Convergent Twist Deformations}{Chiara Esposito, Michael Heins, Stefan Waldmann}\textbf{math.QA}, math-ph, math.FA.

本文为 Drinfeld 的 Universal Deformation Formula (UDF) 在解析向量空间上的收敛性建立了一个函子性框架. 通过将解析向量的阶与 Drinfeld twist 的等度连续条件相匹配, 证明了形式幂级数的收敛性、变形双线性映射的连续性以及其对变形参数 $\hbar$ 的全纯依赖性. 最后, 将理论应用于 Giaquinto 和 Zhang 构造的显式 Drinfeld twists, 验证了等度连续条件并确定了具体表示对应的解析向量空间.


\section{others}
\arxivwithtarget{2602.15941}{On the Jacobian of $\overline{{{\rm Spec}\,\mathbb Z}}$}{Alain Connes, Caterina Consani}\textbf{math.NT}, math.AG, math.QA.

本文探讨了有理数的adele类空间的结构, 特别是其Riemann部分, 将其解释为算术曲线$\overline{\operatorname{Spec} \mathbb Z}$的Picard群的自然幺半群扩张. 作者将adele类空间的元素与带有刚性化数据的无挠秩1阿贝尔群联系起来.


\arxivwithtarget{2602.15944}{Towards a classification of graded unitary ${\mathcal W}_3$ algebras}{Christopher Beem, Harshal Kulkarni}\textbf{hep-th}, math.QA, math.RT.

本文研究了四维幺正性对通过SCFT/VOA对应产生的${\mathcal W}_3$顶点代数的限制. 在$\mathfrak{R}$-滤过是权滤过的假设下, 证明了除$(3,q+4)$极小模型外的中心荷值均与四维幺正性不相容.


\arxivwithtarget{2602.16104}{On the discrete Heine-Shephard problem for four lattice polygons}{Darren Gerrity, Ivan Soprunov}\textbf{math.CO}, math.AG.

本文研究了四个平面格点多边形体积多项式的无平方部分集合, 这与描述具有指定Newton多面体和一般系数的四条曲线在$(\mathbb{C}^*)^2$中的可能相交数问题相关. 作者发现, 对于格点多边形, 经典的Plücker型不等式刻画失效, 并揭示了由格点多边形的混合面积带来的额外算术约束.


\arxivwithtarget{2602.16184}{On Toric Ideals Arising from the Chip-Firing Game}{Rahul Karki}\textbf{math.AC}, math.AG, math.CO, math.RA.

本文研究了在pargraphs(图的一种推广) 上的chip-firing游戏所导出的群和理想. 作者证明了toppling理想成为toric ideal的充分条件, 并构造了其Gröbner基和相关理想的极小胞腔自由分解.


\arxivwithtarget{2602.16248}{Intersections of special cycles on Shimura curves and Siegel Maass forms}{Jan Hendrik Bruinier, Yingkun Li, Martin Möller}\textbf{math.NT}, math.AG.

本文研究了Shimura曲线上特殊循环的交点与Siegel Maass形式之间的联系. 作者证明了交点数的生成级数是某个非全纯Siegel模形式的系数, 并给出了更一般的theta提升的几何解释.


\arxivwithtarget{2602.16452}{On coefficients, potentially abelian quotients, and residual irreducibility of compatible systems}{Gebhard Böckle, Chun-Yin Hui}\textbf{math.NT}, math.AG, math.RT.

本文研究了数域$K$上半单$E$-有理相容系统$\{\rho_\lambda\}$的性质. 作者首先利用伪特征理论定义了每个表示$\rho_\lambda$的代数单值群$G_\lambda$, 并证明了该系统可下降为一个强$E'$-有理相容系统. 其次, 他们证明了$G_\lambda$的最大潜在阿贝尔商在强意义下与$\lambda$无关. 最后, 作为应用, 推广了Patrikis--Snowden--Wiles关于相容系统剩余不可约性的一个结果.


%body_end
    
    %others_begin

    %others_end
\end{document}
