\documentclass[9pt, a4paper]{article}
\usepackage{geometry}        
\geometry{scale=0.75}
\usepackage{setspace}        
%\usepackage{fontspec}        
\usepackage{xeCJK}           
\usepackage{ctex}            
%\usepackage{CJKutf8}         
\usepackage{zhnumber}     
%\usepackage[english,chinese]{babel}
%\usepackage{polyglossia}

\usepackage{amsmath}         
\usepackage{amssymb}         
\usepackage{amsthm}          
\usepackage{mathtools}       
\usepackage{mathrsfs}        
\usepackage{latexsym}        
\usepackage{stmaryrd}        

\usepackage{graphicx}        
\usepackage{tikz}            
\usepackage{tikz-cd}                   

%\usepackage[authoryear]{natbib} 
%\usepackage{etoolbox}        

\usepackage{xcolor}          
\usepackage{hyperref}        
\hypersetup{                 
    colorlinks=true,         
    citecolor=green,         
    urlcolor=red,            
    bookmarksnumbered=true   
}
%\usepackage{cleveref}           
%\usepackage{enumitem}        
%\usepackage{fancyhdr}        
%\usepackage{lmodern}         

%\usepackage{pdfpages} 

\theoremstyle{plain}
\newtheorem{theorem}{定理}
\newtheorem{lemma}[theorem]{引理}
\newtheorem{corollary}[theorem]{推论}
\newtheorem{proposition}[theorem]{命题}

\theoremstyle{definition}
\newtheorem{definition}[theorem]{定义}
\newtheorem{example}[theorem]{示例}
\newtheorem{remark}[theorem]{注记}

\theoremstyle{remark}

\definecolor{arxivblue}{RGB}{0, 102, 204}
\newcommand{\arxivwithtarget}[3]{\hypertarget{#1}{}\href{https://arxiv.org/abs/#1}{#1}, \textbf{\textcolor{arxivblue}{#2}}, #3. }
\newcommand{\arxiv}[3]{\href{https://arxiv.org/abs/#1}{#1}, \textbf{\textcolor{arxivblue}{#2}}, #3. }



\begin{document}
\begin{center}
    \Huge\textbf{不漏arXiv:\today \footnote{本文档由\href{https://github.com/vegetablefj/bluearXiv-ai}{bluearXiv-ai}自动生成. 实际抓取的是``new''页面的数据, 即最近的有所在分类论文变动的一天的数据. This document is automatically generated by \href{https://github.com/vegetablefj/bluearXiv-ai}{bluearXiv-ai}. The data actually captured is from the ``new'' page, that is, the data of the most recent day when there were changes in corresponding subjects.}\footnote{感谢arXiv提供的服务. Thanks for services prodived by arXiv.}\footnote{评论和精选由AI生成, 不代表任何人对论文本身的看法. 精选依赖于论文与给定关键词的匹配度. Comments and selection of good papers are generated by AI, not showing anyone's point of view about those papers. The selection also depends on the matched-degrees between papers and given keywords.}\footnote{下面的计数基于主学科, 不计重数. The following counters are based on main subject, not counting multiplicities.}} 
\end{center}

\begin{center}
    %counter_begin
math.AG: 10

math.RT: 3

math.QA: 0

others: 9

%counter_end   
\end{center}

\setcounter{section}{-1}
\section{精选 Selections}
    %selection_begin
%selection_end

    %body_begin
\section{math.AG}
\arxivwithtarget{2602.16815}{Classifying binary quadratic forms using Clifford invariants}{Soham Mondal, T.E. Venkata Balaji}\textbf{math.AG}, math.AC, math.NT.

本文通过函子性方法, 将秩为二的向量丛上取值于线丛的二次型的相似类, 与关联的广义Clifford代数的零次和一次分量构成的同构类配对联系起来. 作为应用, 该工作推广了Gauss Composition并探讨了与二次代数Picard群的联系.


\arxivwithtarget{2602.16877}{Dimension bounds for relative character varieties on the projective line with three punctures $G=GL(r), O(r), Sp(r)$}{Emmett Lennen}\textbf{math.AG}.

本文研究了在射影线上去除三个点后, 以$G=GL(r), O(r), Sp(r)$为结构群的相对特征簇. 作者利用Simpson的图解法, 对MC-极小特征簇的维数$d>2$给出了一个关于秩$r$的显式线性上界$R(d)$. 通过Katz的中卷积, 任意特征簇都同构于一个满足该上界的簇.


\arxivwithtarget{2602.16904}{Translational surfaces and iterated resultants}{Matthew Weaver}\textbf{math.AG}, math.AC.

本文针对translational surface的隐式化问题, 提出了一种基于iterated homogeneous resultants的新算法. 与现有方法相比, 该算法使用的Sylvester矩阵规模更小, 且在参数化存在ill-behaved basepoints时更具鲁棒性.


\arxivwithtarget{2602.16910}{Webs and smooth components of two column Springer fibers}{Mike Cummings}\textbf{math.AG}.

本文探讨了 $\mathfrak{sl}_k$ 的 web 图与 Springer fiber 之间的联系, 将 Fung 在“两行”情形下的结果推广到了更复杂的“两列”情形. 作者利用 web 清晰地刻画了两列矩形 Springer fiber 的光滑分支及其几何结构, 并证明了这些光滑分支的 Poincaré 多项式在 web 的 dihedral 作用下保持不变.


\arxivwithtarget{2602.16941}{The GKZ hypergeometric $\mathcal D$-module}{Lei Fu}\textbf{math.AG}.

本文从同调函子的角度定义了与$A$-超几何系统密切相关的GKZ超几何$\mathcal D$-模, 并证明了该模是holonomic的, 且在参数化非退化Laurent多项式的Zariski开子集上是一个秩为$n!\mathrm{vol}(\Delta_\infty)$的可积联络.


\arxivwithtarget{2602.16945}{On rational chain connectedness of globally +-regular varieties}{Emre Alp Özavcı, Zsolt Patakfalvi, Kevin Tucker, Joe Waldron, Zheng Xu}\textbf{math.AG}.

本文研究了 globally $+$-regular varieties 的有理链连通性. 证明了在三维及混合特征 (剩余域特征 $p>5$) 情形下, 这类簇是有理链连通的, 并引入了 strongly globally $+$-regular 的概念, 证明了在 $\mathrm{Spec}(\mathbb{Z})$ 的稠密开子集上满足此性质的任意维簇也是有理链连通的.


\arxivwithtarget{2602.17140}{Automorphisms of Smooth Hypersurfaces with Fixed Loci of Codimension at Most Two}{Taro Hayashi, Ryoichi Suzuki}\textbf{math.AG}.

本文研究了射影空间$\mathbb{P}^{n+1}$中光滑超曲面($n\geq2$) 的自同构, 其固定点集的余维数至多为2. 文章探讨了自同构阶数与代数几何性质的关系, 并分析了特定阶数自同构下商空间的有理性.


\arxivwithtarget{2602.17230}{Spectrum, Tjurina spectrum, and Hertling conjecture for singularities of modality $\leq 3$}{Quan Shi, Yang Wang, Huaiqing Zuo}\textbf{math.AG}.

本文计算了trimodal奇点的spectrum并验证了Hertling猜想, 同时定义了Tjurina spectrum并对其提出了广义Hertling猜想, 并对模数$\leq 3$的奇点证明了该猜想.


\arxivwithtarget{2602.17451}{Dimension of fixed loci of diagonalizable groups via algebraic cobordism}{Olivier Haution}\textbf{math.AG}, math.KT.

本文利用代数配边理论研究了 diagonalizable group 在光滑射影簇上的作用, 确定了其不动点集维数所受的来自陈数的所有限制. 作者通过分析等变配边环并构造足够多的显式作用例子, 证明了所获下界的最优性.


\arxivwithtarget{2602.17617}{Generically log smooth families via generators and relations}{Simon Felten}\textbf{math.AG}.

本文针对由生成元与关系式给出的仿射或射影平坦态射, 开发了研究其 log 几何性质的算法工具, 并给出了在 Macaulay2 中的实现. 此外, 文章还讨论了环面交叉概形上 log 光滑结构层的一些结果.


\section{math.RT}
\arxivwithtarget{2602.17197}{On endomorphism algebras of silting complexes over hereditary abelian categories}{Wei Dai, Changjian Fu, Liangang Peng}\textbf{math.RT}.

本文研究了由遗传Abelian范畴上的silting复形的自同态代数构成的类$\mathcal{E}$的封闭性质. 证明了$\mathcal{E}$类在幂等商、幂等子代数和$\tau$-约化操作下是封闭的. 同时, 也证明了包含shod代数在内的其他几个经典代数类在幂等商操作下是封闭的.


\arxivwithtarget{2602.17323}{Iterated mutations of symmetric periodic algebras}{Adam Skowyrski}\textbf{math.RT}.

本文研究了对称周期代数的迭代突变. 作者将A. Dugas用于研究对称代数导出等价对的方法应用于特定情形, 证明了在满足额外假设的条件下, 若顶点$i$对应的单模$S_i$具有周期$d$, 则在该顶点处的突变$\mu_i$具有阶$d-2$.


\arxivwithtarget{2602.17370}{Fukaya categories of orbifold surfaces in representation theory}{Severin Barmeier, Zhengfang Wang}\textbf{math.RT}, math.RA, math.SG.

本文介绍了具有orbifold奇点的曲面的部分缠绕Fukaya范畴. 通过将orbifold曲面$\mathbf S$分解为多边形, 某些分解产生了形式生成元, 从而诱导了$\mathbf S$的导出Fukaya范畴与一个分次结合代数的完美导出范畴之间的三角等价. 这为获得与skew-gentle代数导出等价的结合代数提供了一种几何方法. 文章还提供了一个关于orbifold圆盘的部分缠绕Fukaya范畴的新视角, 该范畴可作为一般orbifold曲面Fukaya范畴的局部模型.


\section{others}
\arxivwithtarget{2602.16770}{From Multipartite Entanglement to TQFT}{Michele Del Zotto, Abhijit Gadde, Pavel Putrov}\textbf{hep-th}, cond-mat.str-el, math-ph, math.QA, quant-ph.

本文探讨了高能物理中多体纠缠与拓扑量子场论(TQFT)之间的联系, 提出了一种关于$(d+1)$-体纠缠与TQFT配分函数之间关系的猜想. 作者以(2+1)维Levin-Wen string-net模型为例验证了这一猜想.


\arxivwithtarget{2602.17075}{$C(SO_q(2n+1)/SO_q(2n-1))$ as iterated torsioned quantum double suspensions of $C(\mathbb{T})$}{Bipul Saurabh}\textbf{math.OA}, math.QA.

本文研究了量子群 $SO_q(2n+1)$ 的齐次空间对应的 $C^*$-代数结构. 作者证明了该代数同构于对环面 $C^*$-代数 $C(\mathbb{T})$ 进行一系列特定扭结 (torsioned) 量子双悬垂 (quantum double suspension) 迭代操作的结果. 这一同构关系表明这些空间的 $C^*$-代数结构不依赖于形变参数 $q$.


\arxivwithtarget{2602.17167}{Non-hyperelliptic modular curves of genus 3}{Enrique González-Jiménez, Roger Oyono}\textbf{math.NT}, math.AG.

本文研究了定义在有理数域$\mathbb{Q}$上的亏格为3的非超椭圆模曲线. 作者给出了此类曲线存在的一个充要条件, 并提出了一个计算其显式方程的算法.


\arxivwithtarget{2602.17179}{Multispecies inhomogeneous $t$-PushTASEP with general capacity}{Arvind Ayyer, Atsuo Kuniba}\textbf{math-ph}, math.CO, math.PR, math.QA.

本文研究了一种具有一般容量的多物种非齐次$t$-PushTASEP. 通过将其Markov矩阵与量子群$U_t(\widehat{sl}_{n+1})$的表示相关联, 给出了稳态概率的矩阵乘积形式.


\arxivwithtarget{2602.17240}{Serre depth and local cohomology}{Antonino Ficarra}\textbf{math.AC}, math.AG, math.CO, math.LO.

本文引入了称为Serre depth的同调不变量, 它分层了Serre条件, 类似于depth分层Cohen-Macaulay性质. 作者研究了该不变量在局部环和分次代数上的性质, 并将其与初始理想、单项式理想的骨架等联系起来.


\arxivwithtarget{2602.17417}{Computing class groups and gonalities of algebraic curves over finite fields}{Maarten Derickx, Kenji Terao}\textbf{math.NT}, math.AG.

本文提出了计算有限域上代数曲线的 divisor class group 和 gonality 的实用算法, 通过引入涉及幂级数展开的预计算步骤, 显著提升了计算效率.


\arxivwithtarget{2602.17435}{A $y$-ification of Khovanov homology}{Taketo Sano}\textbf{math.GT}, math.QA, math.RT.

本文在 Bar-Natan 的 tangle 框架内, 为 Khovanov homology 及其等变版本构造了 $y$-ification, 并定义了 $\mathfrak{sl}_2$ 中元素 $e$ 在其上的作用. 该构造与 Rasmussen 从 HOMFLY--PT homology 到 Khovanov homology 的谱序列相容, 并能区分一些具有相同 Khovanov 和 HOMFLY--PT homology 的 knot 对.


\arxivwithtarget{2602.17549}{Prefactorization algebras for the conformal Laplacian: Central charge and Hilbert Fock space}{Yuto Moriwaki}\textbf{math-ph}, math.DG, math.QA.

本文研究了共形Laplacian的预因子化代数(prefactorization algebra), 特别是其在二维情形下与中心荷(central charge)的关联. 作者证明了该代数在欧氏区域上可通过Green函数与调和函数的对偶空间对称代数自然等同, 并在二维时此等同性的破坏由一个显式调和上链控制. 对于单位圆盘, 该代数结构可稠密嵌入到Hilbert Fock空间中, 这与对数共形场论(logarithmic CFT)的预期一致.


\arxivwithtarget{2602.17661}{Dehn quandles of surfaces and their bounded cohomology}{Pankaj Kapari, Deepanshi Saraf, Mahender Singh}\textbf{math.GT}, math.GR, math.QA.

本文研究了与曲面相关的 Dehn quandles 及其有界上同调. 作者引入了新的 quandle 族作为闭可定向曲面的分类不变量, 并计算了其第二有界 quandle 上同调, 证明其为无穷维的.


%body_end
    
    %others_begin

    %others_end
\end{document}
